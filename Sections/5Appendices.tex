% !TeX root = ../main.tex
\section{Nondimensionalisation of the original LK Equations}
\label{app:LK_nondim}
%
\let\cleardoublepage\origcleardoublepage
%
The LK equations in dimensional form are
%
\begin{gather}
\label{eq:dimensionalised_LK}
\begin{aligned}
\frac{d E}{d s} &= \tfrac{1}{2} G_{\mathrm{M}0}(1 + i \alpha)\,(N - N_N) E 
+ \kappa_0 e^{-i C_p} F(s, \tau_0), \\[0.3em]
\frac{d N}{d s} &= J_0 - \gamma_0 N 
- \left[\Gamma_{\mathrm{M}0} + G_{\mathrm{N}0}(N - N_N)\right] |E|^2,
\end{aligned}
\end{gather}
%
where \(E\) is the slowly varying complex field amplitude and \(N\) the carrier density.
The parameters are defined as follows: 
\(\alpha\) is the linewidth enhancement factor, 
\(\gamma_0\) the carrier decay rate, 
\(\Gamma_{\mathrm{M}0}\) the cavity decay rate, 
\(\kappa_0\) the feedback strength, 
and \(\tau_0\) the external round–trip delay. 
The material response enters through the small-signal gain coefficients 
\(G_{\mathrm{M}0}\) and \(G_{\mathrm{N}0}\), 
while \(J_0\) denotes the pump current and \(N_N\) the inversion at threshold.
%
\par
%
To simplify the equations and reduce the number of parameters, we introduce scaled variables
\[
X = \frac{G_{\mathrm{M}0}}{2 \gamma_p}(N - N_N), 
\qquad 
A = \sqrt{\frac{G_{\mathrm{N}0}}{2 \gamma_0}}\, E,
\]
where \( \gamma_p = \Gamma_{\mathrm{M}0} G_{\mathrm{M}0} / G_{\mathrm{N}0} \) is a rescaled photon lifetime. 
Rescaling time as \( t = \gamma_p s \), the Lang–Kobayashi equations become
\[
\begin{gathered}
\frac{d A}{d t} = (1 + i \alpha)\, A X + \eta e^{-i C_p} F(t,\tau), \\[0.3em]
T \frac{d X}{d t} = P - X - (1 + 2X)|A|^2.
\end{gathered}
\]
Here, the dimensionless parameters are defined as follows. 
The pumping rate above threshold is 
\[
P = \frac{G_{\mathrm{M}0}}{2 \gamma_p \gamma_0}(J_0 - \gamma_0 N_N),
\]
the ratio of photon to carrier lifetime is 
\[
T = \frac{\gamma_p}{\gamma_0},
\]
the normalized feedback strength is 
\[
\eta = \frac{\kappa_0 G_{\mathrm{N}0}}{G_{\mathrm{M}0} \Gamma_{\mathrm{M}0}},
\]
and the rescaled external delay is 
\[
\tau = \gamma_p \tau_0.
\]
%
As a concrete example, consider the dimensionalised parameters  
$
\alpha = 3.5, \; 
\gamma_0 = 1.0 \times 10^9, \; 
\Gamma_{\mathrm{M}0} = 0.55 \times 10^9, \;
\kappa_0 = 25.0 \times 10^9, \; 
\tau_0 = 0.22 \times 10^{-9}, \;
G_{\mathrm{M}0} = 50.0, \; 
G_{\mathrm{N}0} = 0.05, \; 
J_0 = 8.0 \times 10^9, \; 
N_N = 5.0.
$
%
These yield a rescaled photon lifetime \(\gamma_p = 5.5 \times 10^{11}\), giving the nondimensionalised parameters
%
$
\alpha = 3.5, \; 
P = 0.136, \; 
T = 550, \; 
\eta = 0.0455, \; 
\tau = 121,
$
%
which are the values used in Sections~\ref{subsec:FBG_discretised_derivation} and~\ref{subsec:EGM_structure}.  
%
\par
%
The nondimensionalised time unit corresponds to a photon lifetime of \(\tau_p = 1/\gamma_p = 1.82 \times 10^{-12}\,\text{s}\). 
For operation at \(\lambda = 1550\,\text{nm}\), the laser angular frequency is \(\omega_c = 1216\,\text{rad\,ps}^{-1}\). 
It follows that \(\wz\) in the LK equations is dimensionless, normalised by \(\tau_p = 1/\gamma_p = 1.82\,\text{ps}\). 
With this scaling, the angular frequency is nondimensionalised to \(\omega_{\max} = 2211\), which in turn requires \(N \approx 22{,}500\) layers according to~\eqref{eq:wz_approx}.
%
%
\subsection{Discretisation of dimensionalised convolution FBG–LK model}
\label{app:Li_chaos_nondim}
%
We show here how the convolution-based model of Li \textit{et al.} \cite{li2012distributed,li2015chaotic,li2020stable} reduces, under mild and explicitly stated approximations, to the dimensionalised LK form \eqref{eq:dimensionalised_LK} used throughout this paper.
%
Li \textit{et al.} consider
%
\begin{equation} 
    \left\{ 
        \begin{aligned} \frac{d a}{d t} &= \frac{1-i b}{2} \left[\frac{\gamma_{c} \gamma_{n}}{\gamma_{s} \tilde{J}} \tilde{n} - \gamma_{p}\left(|a|^{2}-1\right)\right] a \\ 
            &\hspace{2cm} + \gamma_{c} \xi_{f} e^{i \theta} \left[r(t) e^{-i \Delta \Omega t}\right] * a\!\left(t-\tau_\text{RT}\right) \\ 
            \frac{d \tilde{n}}{d t} &= -\left(\gamma_{s}+\gamma_{n}|a|^{2}\right) \tilde{n} - \gamma_{s} \tilde{J}\left(1-\frac{\gamma_{p}}{\gamma_{c}}|a|^{2}\right) \left(|a|^{2}-1\right) 
        \end{aligned} \right. 
\end{equation}
%
where $r(t)$ is the impulse response associated with the FBG reflectivity and $\Delta\Omega$ is the grating detuning. 
For consistency with \eqref{eq:dimensionalised_LK}, define the notational relabeling
$
a=E,\; \tilde{n}=N,\;
F(t):=\big[r(t)e^{-i\Delta\Omega t}\big]*E\!\left(t-\tau_{\rm RT}\right),\; t=s,
$
 which gives
\begin{equation}
\label{eq:Li_rewritten}
\left\{
\begin{aligned}
\frac{d E}{d s} &= \frac{1-ib}{2}
\left[\frac{\gamma_{c}\gamma_{n}}{\gamma_{s}\tilde{J}}\,N
      -\gamma_{p}\big(|E|^{2}-1\big)\right]E
+ \gamma_{c}\,\xi_{f}\, e^{i\theta} F, \\
\frac{d N}{d s} &= -\big(\gamma_{s}+\gamma_{n}|E|^{2}\big) N
   - \gamma_{s}\tilde{J}\!\left(1-\frac{\gamma_{p}}{\gamma_{c}}|E|^{2}\right)\!\big(|E|^{2}-1\big).
\end{aligned}
\right.
\end{equation}
%
\par
%
We assume the usual near-threshold/weak-saturation ordering used by Li \textit{et al.}:
%
\[
\frac{\gamma_{c}\gamma_{n}}{\gamma_{s}\tilde{J}} \gg \gamma_{p},
\]
so the $-\tfrac{1-ib}{2}\gamma_p(|E|^2-1)E$ term may be dropped in the field equation to leading order. 
Under this ordering, \eqref{eq:Li_rewritten} becomes
\begin{equation}
\label{eq:Li_leading}
\left\{
\begin{aligned}
\frac{d E}{d s} &\approx 
\frac{1}{2}\frac{\gamma_{c}\gamma_{n}}{\gamma_{s}\tilde{J}}\,(1-ib)\,N E
+ \gamma_{c}\xi_f e^{i\theta} F,\\[0.25em]
\frac{d N}{d s} &= -\gamma_s N - \gamma_n|E|^2 N 
- \gamma_s\tilde{J}\!\left(1-\frac{\gamma_p}{\gamma_c}|E|^2\right)\!(|E|^2-1).
\end{aligned}
\right.
\end{equation}
Expanding the carrier equation and grouping powers of $|E|^2$ yields
\begin{align*}
\frac{d N}{d s} 
&= \gamma_s\tilde{J} - \gamma_s N 
  - \Big[\gamma_s\tilde{J}\!\left(1+\tfrac{\gamma_p}{\gamma_c}\right)+\gamma_n N\Big]|E|^2
  + \frac{\gamma_s\gamma_p\tilde{J}}{\gamma_c}|E|^4.
\end{align*}
Neglecting the quartic term $|E|^4$ under weak saturation gives the closed system
\begin{equation}
\label{eq:Li_reduced}
\left\{
\begin{aligned}
\frac{d E}{d s} &= 
\frac{1}{2}\frac{\gamma_{c}\gamma_{n}}{\gamma_{s}\tilde{J}}\,(1-ib)\,N E
+ \gamma_{c}\xi_f e^{i\theta} F,\\
\frac{d N}{d s} &= \gamma_s\tilde{J} - \gamma_s N 
- \Big[\gamma_s\tilde{J}\!\left(1+\tfrac{\gamma_p}{\gamma_c}\right)+\gamma_n N\Big]|E|^2.
\end{aligned}
\right.
\end{equation}
%
\par
%
Comparing \eqref{eq:Li_reduced} with \eqref{eq:dimensionalised_LK}, we obtain the one-to-one parameter correspondence
%
$
b=-\alpha,\; \theta=-C_p,\; 
G_{\mathrm{M}0}=\gamma_c\gamma_n/\gamma_s\tilde{J},\;
\kappa_0=\gamma_c\xi_f,\;
J_0=\gamma_s\tilde{J},\; 
\gamma_0=\gamma_s,\; 
\Gamma_{\mathrm{M}0}=\gamma_s\tilde{J}\!\left(1+\gamma_p/\gamma_c\right),\;
G_{\mathrm{N}0}=\gamma_n,
$
%
and the external delay $\tau_0=\tau_{\rm RT}$. 
The convolution kernel $r(t)e^{-i\Delta\Omega t}$ plays the role of the (dimensioned) impulse response used in \eqref{eq:dimensionalised_LK}, with $\Delta\Omega$ corresponding to the grating detuning.
%
\par
%
In summary, under the standard weak-saturation ordering and after neglecting the quartic $|E|^4$ term, the convolution-based equations of \cite{li2012distributed,li2015chaotic,li2020stable} reduce exactly to the dimensionalised LK form \eqref{eq:dimensionalised_LK}, with the above parameter identifications and with the feedback written as a time-domain convolution against the FBG impulse response.
This establishes the formal equivalence used for the comparisons in the main text.
%
%
\section{Discretisation of multiple external-cavity reflections in the FBG LK model}
\label{app:EC_multiple}
%
A key extension of the single-reflection Lang–Kobayashi model is to include multiple reflections within the EC. 
In this case, the reflected field can be written as
\begin{equation*}
    E_r(t) = E(t) + \frac{R_2^2 - 1}{R_2^2} \sum_{n=1}^\infty (-R_2 R)^n e^{-i n C_p} E(t-n \tau),
\end{equation*}
%
where $R_2$ is the reflectivity of the front facet of the laser, $R$ the external mirror reflectivity, and $C_p$ the cavity phase. 
Truncating this series to first order recovers the familiar single-reflection feedback term. 
Indeed, expanding the logarithm in the feedback term gives
%
\begin{equation*}
    \begin{aligned}
    E(t) \ln{\!\left(\frac{E_r(t)}{E(t)}\right)} 
    &\approx E(t)\!\left(\frac{\tfrac{R_2^2 - 1}{R_2^2}(-R_2 R)e^{-i C_p} E(t-\tau)}{E(t)}\right) \\
    &= \eta e^{-i C_p} E(t-\tau),
    \end{aligned}
\end{equation*}
%
as expected.
%
%
\subsubsection*{Multiple-reflection formulation}
%
Naumenko \textit{et al.} \cite{naumenko2003characteristics,naumenko2004slow} generalised the LK equations to explicitly incorporate this logarithmic multiple-reflection term:
%
\begin{equation*} 
    \left\{\begin{aligned}
         \frac{d E}{d s} &= \frac{1}{2} \tau_\text{in} \Gamma G_N(1+i \alpha)\left(N-N_{\text {th }}\right) E + \ln \left(\frac{\bar{F}}{E}\right) E \\ 
        \frac{d N}{d s} &= \frac{\tau_{\text{in}}\left(J-\bar{J}_0\right)}{e V}-\frac{\tau_\text{in}}{\tau_e} N-\left[\frac{\tau_\text{in}}{\Gamma \tau_{ph}}+G_N \tau_\text{in}\left(N-N_{\text{th}}\right)\right]|E|^2 
    \end{aligned}\right. 
\end{equation*}
%
with
%
\begin{gather} 
    \begin{aligned} 
    \bar{F}(t) &= E+\frac{(-1)}{2 \pi} \frac{\left(1-r_2^2\right)}{r_2^2} \sum_{n=1}^{\infty} e^{-i \omega_0 n \tau} \times \\ 
    &\int_{-\infty}^{+\infty} d \omega\left(-r_2 r_B(\omega)\right)^n e^{i \omega(t-n \tau)} \int_{-\infty}^{+\infty} d t^{\prime} E\left(t^{\prime}\right) e^{-i \omega t^{\prime}} 
    \end{aligned} 
\end{gather}
%
%
\subsubsection*{Reduction to the standard LK form}
%
By neglecting gain compression ($\epsilon \approx 0$) and thermal detuning ($k=0$), and linearising the spontaneous emission term, the equations reduce to
%
\begin{equation*}
\left\{
\begin{aligned}
\frac{d E}{d t} &= \tfrac{1}{2} \Gamma G_N(1+i \alpha)(N-N_\text{th}) E 
+ \frac{1}{\tau_\text{in}} \ln \!\left(\tfrac{\bar{F}(t)}{E}\right) E, \\
\frac{d N}{d t} &= \frac{J-\bar{J}_0}{e V}-\frac{N}{\tau_0}
-\Big[\tfrac{1}{\Gamma \tau_{ph}} + G_N(N-N_\text{th})\Big]|E|^2.
\end{aligned}
\right.
\end{equation*}
%
Rescaling time by $t=\tau_\text{in} s$ yields
%
\begin{equation*}
\left\{
\begin{aligned}
\frac{d E}{d s} &= \tfrac{1}{2} \tau_\text{in} \Gamma G_N(1+i \alpha)(N-N_{\text{th}}) E + \ln \!\left(\tfrac{\bar{F}}{E}\right) E, \\
\frac{d N}{d s} &= \tfrac{\tau_{\text{in}}(J-\bar{J}_0)}{e V}-\tfrac{\tau_\text{in}}{\tau_e} N
-\Big[\tfrac{\tau_\text{in}}{\Gamma \tau_{ph}}+G_N \tau_\text{in}(N-N_{\text{th}})\Big]|E|^2,
\end{aligned}
\right.
\end{equation*}
%
which matches the dimensionalised LK form \eqref{eq:dimensionalised_LK} when identifying
%
$
\alpha = \alpha, \; 
N_N = N_{\text{th}}, \; 
G_\text{M0} = \tau_\text{in} \Gamma G_N, \; 
G_\text{N0} = G_N \tau_\text{in}, \;
J_0 = \tau_\text{in}(J-\bar{J}_0)/e V, \;
\gamma_0 = \tau_\text{in}/\tau_e, \; 
\Gamma_\text{M0} = \tau_\text{in}/\Gamma \tau_{ph}, \; 
\tau_0 = 2 L_\text{ext} n_\text{ext}/c \tau_\text{in}
$.
%
\par
%
Finally, the multiple-reflection feedback term reduces in the single-reflection limit to
%
\begin{equation*}
    E \ln \left(\tfrac{\bar{F}}{E}\right) \approx \frac{1-R_2^2}{R_2} e^{-i C_p} F,
\end{equation*}
%
so that
%
\begin{equation*}
    \kappa_0 = \frac{1-R_2^2}{R_2} R.
\end{equation*}
%
\par
%
Thus the Naumenko model and the standard LK formulation are equivalent up to the choice of effective feedback strength, completing the connection between multiple- and single-reflection descriptions.
%
%
subsection{Derivation of Variational equations for the Discretised reflection FBG LK Equations}
\label{app:lyapunov}
%
Beginning with the discretised reflections model, consisting of the complex valued electric field variable $E$ and real valued carrier density variable $N$ \eqref{eq:LK_discretised}, the variational equations are defined as
%
\begin{equation}
    \frac{d}{d t}\binom{\delta E}{\delta N} = \mathbf{J}(\underline{F})\binom{\delta E}{\delta N} + \sum_{k=1}^{N}\mathbf{J}_{\tau_{k}}(\underline{F})\binom{\delta E_{\tau_{k}}}{\delta N_{\tau_{k}}}
\end{equation}
%
where $\underline{F}(t)=(\dot{E}(t), \dot{N}(t))^T$, which results in
%
\begin{gather}
\begin{aligned}
    \label{eq:variational}
    \dot{\delta E}(t) &= (1+i \alpha)[\delta E \; N + E \; \delta N]\\
                &\hspace{1cm} + \eta(1-t) e^{-i C_p} \sum_{k=0}^{N-1} t^{k} e^{-i \wB \dt} \delta E\left(t-\tau_{k}\right) \\
    \dot{\delta N}(t) &=-\frac{1}{T}\left[2(1+2 N) \text{Re}\left\{\overline{E} \; \delta E\right\} + (1+2|E|^{2} ) \delta N \right].
\end{aligned}
\end{gather}
%
These variational equations then can be used to calculate Lyapunov spectra as done in this work using the method outlined in \cite{farmer1982chaotic}.
