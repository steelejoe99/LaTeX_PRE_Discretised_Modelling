% !TeX root = ../main.tex
\section{Nondimensionalisation of the original LK Equations}
\label{sec:LK_nondim}
%
\let\cleardoublepage\origcleardoublepage
%
To model the dynamics of a semiconductor laser with optical feedback from an external cavity, we use the well-known \textbf{Lang-Kobayashi equations} \cite{lang1980external}. These delay differential equations describe how the complex electric field \( E \) of the laser and the carrier inversion \( N \) evolve in time under the influence of both intrinsic gain mechanisms and delayed feedback. In dimensionalised form, the equations are written as:
%
\begin{gather}
\label{eq:dimensionalised_LK}
\begin{aligned}
\frac{d E}{d s} &= \frac{1}{2} G_{\mathrm{M}0}(1 + i \alpha) \left[N - N_N\right] E + \kappa_0 e^{-i C_p} F(s, \tau_0)\\
\frac{d N}{d s} &= J_0 - \gamma_0 N - \left[\Gamma_{\mathrm{M}0} + G_{\mathrm{N}0}(N - N_N)\right] |E|^2.\\
\end{aligned}
\raisetag{-1.5em} % adjust as needed
\end{gather}
%
The physical parameters in these equations are defined as follows:
\begin{itemize}
  \item \( \alpha \): linewidth enhancement factor
  \item \( \gamma_0 \): carrier decay rate
  \item \( \Gamma_{\mathrm{M}0} \): cavity decay rate
  \item \( \kappa_0 \): feedback strength
  \item \( \tau_0 \): external round-trip delay
  \item \( G_{\mathrm{M}0}, G_{\mathrm{N}0} \): small-signal gain coefficients
  \item \( J_0 \): pump current
  \item \( N_N \): inversion at threshold
\end{itemize}
%
\par
%
To simplify the equations and reduce the number of parameters, we introduce scaled variables:
%
\[
X = \frac{G_{\mathrm{M}0}}{2 \gamma_p}(N - N_N), \quad A = \sqrt{\frac{G_{\mathrm{N}0}}{2 \gamma_0}} E,
\]
where \( \gamma_p = \Gamma_{\mathrm{M}0} G_{\mathrm{M}0} / G_{\mathrm{N}0} \) is a rescaled photon lifetime. Rescaling time via \( t = \gamma_p s \), the Lang-Kobayashi equations become:
%
\[
\begin{gathered}
\frac{d A}{d t} = (1 + i \alpha) A X + \eta e^{-i C_p} F(t,\tau) \\
T \frac{d X}{d t} = P - X - (1 + 2X)|A|^2
\end{gathered}
\]
%
The transformed parameters are defined as:
%
\begin{itemize}
  \item \( P = \frac{G_{\mathrm{M}0}}{2 \gamma_p \gamma_0}(J_0 - \gamma_0 N_N) \): pumping rate above threshold
  \item \( T = \frac{\gamma_p}{\gamma_0} \): ratio of photon to carrier lifetime
  \item \( \eta = \frac{\kappa_0 G_{\mathrm{N}0}}{G_{\mathrm{M}0} \Gamma_{\mathrm{M}0}} \): normalized feedback strength
  \item \( \tau = \gamma_p \tau_0 \): rescaled external delay
\end{itemize}
%
For example, dimensionalised parameter values $\alpha = 3.5, \, \gamma_0 = 1.0 \times 10^9, \, \Gamma_{\mathrm{M}0} = 0.55 \times 10^9,\,
\kappa_0 = 25.0 \times 10^9, \, \tau_0 = 0.22 \times 10^{-9}, \, G_{\mathrm{M}0} = 50.0, \, G_{\mathrm{N}0} = 0.05, \, J_0 = 8.0 \times 10^9, \,
N_N = 5.0$ result in $\gamma_p = 5.5 \times 10^{11}$ and nondimensionalised parameter values $\a = 3.5, \, P = 0.136, \, T = 550, \, \eta = 0.0455, \, \tau = 121$.
%
It is noted that one can get from this nondimensionalised time to seconds by multiplying by the factor $\tau_p = 1/\gamma_p = 1.82 \times 10^{-12}\,\text{s}$.
%
%
\section{Convolution FBG LK model equivalence}
\label{sec:Li_chaos_nondim}
%
The form of the equations analysed by Li \textit{et. al} \cite{li2012distributed,li2015chaotic,li2020stable} are given by
%
\begin{equation}
\left\{
\begin{aligned}
\frac{d a}{d t} &= \frac{1-i b}{2}
   \left[\frac{\gamma_{c} \gamma_{n}}{\gamma_{s} \tilde{J}} \tilde{n}
         - \gamma_{p}\left(|a|^{2}-1\right)\right] a \\
&\hspace{2cm} + \gamma_{c} \xi_{f} e^{i \theta}
   \left[r(t) e^{-i \Delta \Omega t}\right] * a\!\left(t-\tau_\text{RT}\right) \\
\frac{d \tilde{n}}{d t} &= -\left(\gamma_{s}+\gamma_{n}|a|^{2}\right) \tilde{n}
   - \gamma_{s} \tilde{J}\left(1-\frac{\gamma_{p}}{\gamma_{c}}|a|^{2}\right)
   \left(|a|^{2}-1\right)
\end{aligned}
\right.
\end{equation}

%
we first make a change of notation:
%
\begin{equation*}
\left\{\begin{aligned}
a & =E \\
\tilde{n} & =N \\
F(t) & =\left[r(t) e^{-i \Delta \Omega t}\right] * a\left(t-\tau_\text{RT}\right) \\
t & =s
\end{aligned}\right.
\end{equation*}
%
Resulting in
%
\begin{equation*}
\left\{\begin{array}{l}
\displaystyle \frac{d E}{d s}=\frac{1-ib}{2}\left[\frac{\gamma_{c} \gamma_{n}}{\gamma_{s} \tilde{J}} N-\gamma_{p}\left(|E|^{2}-1\right)\right] E+\gamma_{c} \xi_{f} e^{i\theta} F(s) \\
\displaystyle \frac{d N}{d s}=-\left(\gamma_{s}+\gamma_{n}|E|^{2}\right) N-\gamma_{s} \tilde{J}\left(1-\frac{\gamma_{p}}{\gamma_{c}}|E|^{2}\right)\left(|E|^{2}-1\right)
\end{array}\right.
\end{equation*}
%
% Now, rescaling time by $s=\gamma_{c} t$
% %
% \begin{equation*}
% \left\{\begin{array}{l}
% \displaystyle \frac{d E}{d s}=\frac{1-ib}{2}\left[\frac{\gamma_{n}}{\gamma_{s} \tilde{J}} N-\frac{\gamma_{p}}{\gamma_{c}}\left(|E|^{2}-1\right)\right] E+\xi_{p} e^{i\theta} F(s) \\
% \displaystyle \frac{d N}{d s}=-\left(\frac{\gamma_{s}}{\gamma_{c}}+\frac{\gamma_{n}}{\gamma_{c}}|E|^{2}\right) N-\frac{\gamma_{s} \tilde{J}}{\gamma_{c}}\left(1-\frac{\gamma_{p}}{\gamma_{c}}|E|^2\right)\left(|E|^{2}-1\right)
% \end{array}\right.
% \end{equation*}
% %
Focusing on the first equation, we use:
%
\begin{equation*}
    \frac{\gamma_{n}\gamma_{c}}{\gamma_{s} \tilde{J}} \gg \gamma_{p}
\end{equation*}
%
to obtain:
%
\begin{equation*}
    \frac{d E}{d s} \approx \frac{1}{2} \frac{\gamma_n \gamma_c}{\gamma_{s} \tilde{J}}(1-ib) N E+ \gamma_c \xi_{f} e^{i\theta} F(s)
\end{equation*}
%
Then, for the second equation,
%
\begin{align*}
\displaystyle \frac{d N}{d s} &= -\gamma_s N - \gamma_n|E|^{2} N - \gamma_s \tilde{J}|E|^2 + \frac{\gamma_s \gamma_p \tilde{J}} {\gamma_{c}}|E|^{4}\\
&\hspace{3.2cm} + \gamma_{s} \tilde{J} - \frac{\gamma_{s} \gamma_{p} \tilde{J}}{\gamma_{c}}|E|^{2}\\
&\hspace{-0.7cm} =\gamma_{s} \tilde{J}-\gamma_{s} N-\left[\gamma_{s} \tilde{J}\left(1+\frac{\gamma_{p}}{\gamma_{c}}\right)+\gamma_{n} N\right]|E|^{2}+\frac{\gamma_{s} \gamma_{p} \tilde{J}}{\gamma_{c}}|E|^{4}
\end{align*}
%
Neglecting the $|E|^{4}$ term, the system can therefore be written as:
%
\begin{equation*}
\left\{\begin{array}{l}
\displaystyle \frac{d E}{d s}=\frac{1}{2} \frac{\gamma_{n}\gamma_c}{\gamma_{s} \tilde{J}}(1-ib) N E+\gamma_c\xi_{f} e^{i\theta} F(s) \\ 
\displaystyle \frac{d N}{d s}=\gamma_{s} \tilde{J}-\gamma_{s} N-\left[\gamma_{s} \tilde{J}\left(1+\frac{\gamma_{p}}{\gamma_{c}}\right)+\gamma_{n} N\right]|E|^{2}\end{array}\right.
\end{equation*}
%
which is precisely the form of \eqref{eq:dimensionalised_LK} for
%
$$
\begin{aligned}
& b = -\a, \; \theta=-C_{p}, \; G_\text{M0}=\frac{\gamma_{n}}{\gamma_{s} \tilde{J}}, \; \kappa_{0}=\gamma_{c} \xi_{f}, \\
& J_{0}=\gamma_{s} \tilde{J}, \; \gamma_{0}=\gamma_{s}, \; \Gamma_\text{M0}=\gamma_{s} \tilde{J}\left(1+\frac{\gamma_{p}}{\gamma_{c}}\right), \; G_\text{N0}=\gamma_{n}
\end{aligned}
$$
%
%
\section{Multiple EC reflection FBG LK model conversion}
\label{sec:multiple_EC_nondim}
%
The form of the LK equations studied by Naumenko \textit{et al.} \cite{naumenko2003characteristics,naumenko2004slow} is given by
\begin{equation}
\left\{
\begin{aligned}
\frac{d E}{d t} &= \left[\frac{1}{2} \Gamma G_N \big\{ i \alpha\left(N-N_{\text {th }}\right)+g(N, I) \big\} \right. \\
                &\left.\hspace{2cm} - \frac{1}{2 \tau_\mu}+\frac{1}{\tau_\text{in}} \ln \left(\frac{\bar{F}(t)}{E}\right)+i \Delta \omega_0\right] E\\
\frac{d N}{d t} &= \frac{I}{e V}-\frac{N}{\tau_N(N)}-G_N \, g(N, I) I
\end{aligned}
\right.
\end{equation}
%
where $g(N, I)$ is the nonlinear gain, $N / \tau_N(N)$ is the spontaneous emission rate, $\Delta \omega_0$ is the frequency chirp due to thermal effects when the injection current is changed, and,
%
\begin{gather}
\begin{aligned}
\bar{F}(t) &=  E+\frac{(-1)}{2 \pi} \frac{\left(1-r_2^2\right)}{r_2^2} \sum_{n=1}^{\infty} e^{-i \omega_0 n \tau} \times \\
            &\int_{-\infty}^{+\infty} d \omega\left(-r_2 r_B(\omega)\right)^n e^{i \omega(t-n \tau)} \int_{-\infty}^{+\infty} d t^{\prime} E\left(t^{\prime}\right) e^{-i \omega t^{\prime}}
\end{aligned}
\end{gather}
%
Assuming a negligible gain compression factor and thermal frequency detuning (by setting $\epsilon \approx 0$ and $k=0$, respectively, following the expressions in \cite{naumenko2003characteristics}), and linearising the spontaneous emission rate, these equations can be rewritten as
%
\begin{equation*}
\left\{\begin{aligned}
\frac{d E}{d t} &= \frac{1}{2} \Gamma G_N(1+i \alpha)\left(N-N_\text{th}\right) E + \frac{1}{\tau_\text{in}} \ln \left(\frac{\bar{F}(t)}{E}\right) E 
\\
\frac{d N}{d t} &= \frac{J-\bar{J}_0}{e V}-\frac{N}{\tau_0}-\left[\frac{1}{\Gamma \tau_{p h}} + G_N\left(N-N_\text{th}\right)\right]|E|^2
\end{aligned}\right.
\end{equation*}
%
Now, rescaling time by $t=\tau_\text{in} s$, one obtains
%
\begin{equation*}
\left\{\begin{aligned}
\frac{d E}{d s} &= \frac{1}{2} \tau_\text{in} \Gamma G_N(1+i \alpha)\left(N-N_{\text {th }}\right) E + \ln \left(\frac{\bar{F}(s)}{E}\right) E
\\
\frac{d N}{d s} &= \frac{\tau_{\text{in}}\left(J-\bar{J}_0\right)}{e V}-\frac{\tau_\text{in}}{\tau_e} N-\left[\frac{\tau_\text{in}}{\Gamma \tau_{ph}}+G_N \tau_\text{in}\left(N-N_{\text{th}}\right)\right]|E|^2
\end{aligned}\right.
\end{equation*}
%
Which is precisely the form \eqref{eq:dimensionalised_LK} when replacing the multiple EC reflection term $E \ln \left( \bar{F}(s) / E \right)$ with $\kappa_{0,\mathrm{eff}} \, e^{-i C_p} F(s)$ for
%
\[ \alpha = \alpha , \quad N_N = N_{\text{th}}, \quad G_\text{M0} = \tau_\text{in} \Gamma G_N, \quad G_\text{N0} = G_N \tau_\text{in} 
\]
\[
J_0 = \frac{\tau_\text{in}(J-\bar{J}_0)}{e V}, \quad \gamma_0 = \frac{\tau_\text{in}}{\tau_e}, \quad \Gamma_\text{N0} = \frac{\tau_\text{in}}{\Gamma \tau_\text{ph}}, \quad \tau_0 = \frac{2 L_\text{ext} n_\text{ext}}{c \tau_\text{in}} 
\]
%
where:
%
\begin{equation*}
\tau_\text{ph}=\frac{1}{c \alpha_\text{in} / n_g - 1 / \tau_\text{in} \ln \left(R_1 R_2\right)} \text{ and } N_{\text{th}} = N_0+\frac{1}{\Gamma G_N \tau_\text{ph}} 
\end{equation*}
%
$\k_0 = \k_{0,\mathrm{eff}} R$ can be determined by reducing the multiple EC reflection feedback term as a single reflection,
%
\begin{equation*}
E \ln \left( \frac{\bar{F}(s)}{E} \right) \overset{n=1}{\approx} \frac{1-R_2^2}{R_2}e^{-iC_p} \mathcal{F}^{-1}\big[ \rho(\w) \mathcal{F}\left[ E(s-\tau_0) \right](\w) \big](s)
\end{equation*}
%
which is precisely in our notation
%
\begin{equation*}
E \ln \left(\frac{\bar{F}(s)}{E}\right)=\frac{1-R_2^2}{R_2} e^{-i C_p} F(s)
\end{equation*}
%
Therefore, in the single reflection case:
%
\begin{equation*}
    \k_0=\frac{1-R_2^2}{R_2} R
\end{equation*}
%
Then, we can convert to the non-dimensionalised model following the transformations in Appendix~\ref{sec:LK_nondim}.
%
%
\section{Derivation of Variational equations for the Discretised reflection FBG LK Equations}
\label{app:lyapunov}
%
Beginning with the discretised reflections model, consisting of the complex valued electric field variable $E$ and real valued carrier density variable $N$. 
\begin{align*}
\dot{E}&=(1+i \alpha) N E+\eta(1-t) e^{-i C_p} \sum_{k=0}^{N-1} t^{k} e^{-i \omega_B \delta \tau} E\left(t-\tau_k \right) \\
\dot{N}&=\frac{1}{T}\left[P-N-(1+2 N)|E|^{2}\right]
\end{align*}
%
The variational equations are defined as
%
\begin{equation}
    \frac{d}{d t}\binom{\delta E}{\delta N} = \mathbf{J}(\underline{F})\binom{\delta E}{\delta N} + \sum_{k=1}^{N}\mathbf{J}_{\tau_{k}}(\underline{F})\binom{\delta E_{\tau_{k}}}{\delta N_{\tau_{k}}}
\end{equation}
%
where $\underline{F}(t)=(\dot{E}(t), \dot{N}(t))^T$, which results in
%
\begin{gather}
\begin{aligned}
    \dot{\delta E}(t) &= (1+i \alpha)[\delta E \; N + E \; \delta N]\\
                &\hspace{1cm} + \eta(1-t) e^{-i C_p} \sum_{k=0}^{N-1} t^{k} e^{-i \wB \dt} \delta E\left(t-\tau_{k}\right) \\
    \dot{\delta N}(t) &=-\frac{1}{T}\left[2(1+2 N) \text{Re}\left\{\overline{E} \; \delta E\right\} + (1+2|E|^{2} ) \delta N \right]
\end{aligned}
\end{gather}
