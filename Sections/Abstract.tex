% !TeX root = ../main.tex

\begin{abstract}
Semiconductor lasers are compact, efficient light sources widely used in optical communications, and their sensitivity to external feedback makes them rich systems for studying nonlinear dynamics. The Lang-Kobayashi (LK) equations are the standard tool for modelling lasers subject to external feedback from a regular mirror. A key reflective component in optical systems is the fiber Bragg grating (FBG)—a periodic optical fiber refractive index variation—leveraged for its precise spectral control and all-fiber compatibility. When the external feedback comes from an FBG, present modelling requires a computationally expensive convolution term, which provides limited analytical insight into the system’s behaviour. We present a novel modelling approach that approximates FBG feedback by a sum of discrete delay terms. Critically, this avoids the need for numerical convolution while preserving the essential physics. This enables detailed analysis of the laser’s mode structure, stability regimes, and bifurcation structure in the spirit of that for the ‘classic’ LK equations. In this way, our work provides a foundation for deeper theoretical study of semiconductor lasers subject to technologically relevant types of FBG feedback, bridging the gap between numerical simulation and analytical understanding.
\end{abstract}
