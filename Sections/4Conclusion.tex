% !TeX root = ../main.tex
\section*{Conclusions and future developments}
\label{sec:conclusions}
%
While a low reflectivity $R$ does provide the desired transmission zeros, the relative errors in both total reflectivity $R_\text{error}$ and delay $\tau_\text{error}$ are larger compared to larger total reflectivities. 
This is demonstrated in Figure~\ref{fig:NR_Selection}, which visualises the combined relative errors $R_\text{error}$ and $\tau_\text{error}$ using the Euclidean metric $\| R_\text{error}, \tau_\text{error} \|_{_2} = \sqrt{R_\text{error}^2 + \tau_\text{error}^2}$. 
For the choice $N=10$, a total reflectivity $R_\text{approx} = 0.75$ would provide a lower combined error of 4\% as shown in (a) compared to the $R_\text{approx} = 0.1$ which has a combined error of 19\%. 
As the ability to effectively describe and analyse transmission zeros of FBGs is of great interest, this increase in combined error, which is still relatively low from a modelling perspective, is a an acceptable trade-off.
%
\begin{figure}
    \centering
    
    \includegraphics[width=0.95\linewidth]{Images/NR_Selection.pdf}
    
    \caption{Combined relative errors of FBG total reflectivity $R_\text{error}$ and delay $\tau_\text{error}$ as a function of the grating number $N$ and normalised refractive index variation $\dn$. 
    Level sets of constant total reflectivity $R_\text{approx}$ are overlayed in white. 
    In (a), the square marker at $(N,\dn) = (10, 0.07)$, corresponding to an $R_\text{approx} = 0.75$ has a combined error of 4\% while in (b), the marker at $(N,\dn) = (10, 0.0055)$, 
    corresponding to an $R_\text{approx} = 0.1$ has a combined error of 19\%.}
    
    \label{fig:NR_Selection}
\end{figure}
%