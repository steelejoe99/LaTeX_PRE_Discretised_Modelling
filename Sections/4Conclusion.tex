% !TeX root = ../main.tex
\section{Conclusions and future developments}
\label{sec:conclusions}
%
\subsection{Discussion}
\label{subsec:discussion}
%
In this work, we set out to replace the analytically unwieldly convolution-based descriptions of FBG feedback with a formulation that couples directly to grating design parameters while preserving the DDE structure of the LK equations.
The discretised-reflection model developed here achieves exactly that.
By representing a uniform FBG as a finite stack of identical layers, each contributing a delayed, phase-shifted replica of the electric field, the feedback term reduces to a finite sum of discrete delays.
This describes the system within DDE framework, opening the door to closed-form mode equations, rigorous continuation, and variational analyses that are not accessible in the convolution setting.
%
\par
%
A central ingredient is the introduction of layer-by-layer effective transmission and reflection coefficients, $r$ and $t$, chosen so that the summed delayed contributions reproduce the known total reflectivity of a uniform grating.
These coefficients are effective rather than Fresnel: they contain the cumulative action of many internal interfaces while satisfying the intuitive balance $r+t=1$.
This lead to the simple and transparent relation $t = (1-R)^{1/N}$ which links device-level targets (overall reflectivity $R$) to the strength and weighting of the discrete feedback terms.
Beyond reflectivity, the model also reproduces the effective delay of the grating with high accuracy across parameter ranges of interest, so both amplitude and timing benchmarks are met within a single compact parameterisation.
%
\par
%
Formally, the discretised impulse response is an exponentially weighted train of delta functions with phase increments that encode propagation through successive layers.
Substituting this impulse response into the LK equations yielded a feedback term that is a finite sum over delayed fields 
This construction preserves the DDE form and, in the plane-mirror limit $t\rightarrow 0$ recovers the standard COF feedback term, providing a clean bridge between the mirror and grating cases.
%
\par
%
On this foundation we derived the EGMs: steady states of the discretised system analogous to ECMs in the mirror case and EFMs in the Lorentzian-filter case.
We obtained an implicit equation for the EGM frequencies $\w_s$ expressed the EGM envelope in closed form, and recovered the COF ECM relation in the plane-mirror limit.
Critically, the EGM envelope $f_\text{env}$ acts as a counterpart of the grating’s reflection spectrum: its zeros, lobes, and bandwidth reproduce the spectral structure that the grating imposes.
This gives a direct spectral interpretation of the time-domain model and supplies a natural diagnostic for design and analysis.
%
\par
%
We then showed how practical parameter choices render the model both physically faithful and computationally tractable.
To resolve reflection zeros sharply one should use $R\ll1$ and fold overall feedback into an effective rate $\eta R$.
With this condition on $R$, the reflection-zero spacing obeys \eqref{eq:wz_approx} and as a result, the bandwidth can be fixed by keeping the product $N\dt$ (equivalently, the grating length) constant.
This decouples numerical tractability from optical realism: one can work with modest $N$ while preserving the correct spectral window, provided $N$ exceeds the simple lower bounds given by \eqref{eq:Nmin} that prevents artefacts from repeated lobes.
To this end, we demonstrated that EGM structures remain essentially unchanged when 
%
\par
%
We extended these analytic expressions by seeking intersections of $f_\text{env}$ to classify and count EGM components.
Continuation of the coupled conditions \eqref{eq:EGM_components,eq:EGM_components_fold} produces global bifurcation diagrams in the $(\wz,wB)$-plane.
The boundary between one and two components assumes a cusp-shaped curve remarkably similar to that known for FOF feedback, while additional fold curves accumulate and terminate at cusps to create regions with three or more components.
We identified “mixed” components (containing neither the free-running frequency nor the Bragg frequency) that have no analogue in the Lorentzian-filter case.
We further quantified how too-small $N$ distorts inner fold curves and mapped the resulting invalid regions; the predicted $N_\text{min}$ bounds given by \eqref{eq:Nmin} accurately capture the onset of such artefacts.
%
\par
%
We validated the spectral side of the model by comparing EGM structures to those obtained with the multiple-reflection formulation of Naumenko \textit{et al.}
Despite their more elaborate feedback representation, our discretised model reproduces the locations of EGM branches, the maximal-gain mode, and the organisation of solutions across bandwidth and detuning with excellent accuracy.
Distortions along the intensity axis, attributable to gain-suppression effects absent in our simplified equations, do not affect the spectral agreement or number of modes.
%
\par
%
Turning to dynamics, we connected stability fluctuations to the alignment of feedback zeros with intrinsic laser frequencies.
Specifically, overlap with the free-running frequency and overlap with the RO frequency. 
In the first case, zeros suppress the central lasing resonance, creating stability windows whose boundaries mirror the inverse of the reflection profile.
In the second, zeros suppress the RO sidebands, producing stability peaks at grating lengths that satisfy \eqref{eq:RO_stability}.
Both mechanisms, previously identified in convolution-based simulations, are recovered here within a framework that permits precise Hopf continuation and direct comparison to EGM stability.
%
\par
%
Comparisons against two-parameter dynamical mappings from Li \textit{et al.} show close agreement in the organisation of steady, periodic, and complex dynamics across the detuning–feedback plane.
Our formulation sharpens these maps by overlaying Hopf curves of steady EGMs and, crucially, by enabling computation of the full Lyapunov spectrum via explicit variational equations.
This makes it possible to diagnose instability growth rates, distinguish quasi-periodic from chaotic regimes, and estimate attractor dimension via the Kaplan–Yorke formula, capabilities that are either impractical or only partially applied in convolution studies.
In parameter regions previously labelled “chaotic”, we can state precisely where positive exponents occur, where tori bound higher-period windows, and how these sets organise around codimension-2 points.
%
\par
%
Finally, investigated the RO-suppression mechanism in the context closest to our equations (\Skenderas \textit{et al.}).
Tracking Hopf loci across grating length and feedback phase, our continuation results reproduce the stability peaks predicted by RO-reflection zero alignment and match the reported LLE-based stability boundaries, while providing higher precision and clearer interpretation.
In an extended detuning–feedback parameter sweep, we identified double-Hopf, fold–Hopf, and double-torus points and the torus and fold curves they generate.
These organising centres explain how stability islands, quasi-periodic bands, and transition layers are laid out, insight that is difficult to obtain from time-series diagnostics alone.
%
%
\subsection{Limitations and Outlook}
\label{subsec:outlook}
%
A first limitation of the present study lies in the trade-off between reflectivity and modelling accuracy. 
While choosing a small total reflectivity $R$ ensures the presence of sharp transmission zeros, it also amplifies the relative errors in both the total reflectivity ($R_\text{error}$) and effective delay ($\tau_\text{error}$). 
Figure~\ref{fig:NR_Selection} illustrates this trade-off by combining the two errors into a single Euclidean metric, 
$\| R_\text{error}, \tau_\text{error} \|_{2} = \sqrt{R_\text{error}^2 + \tau_\text{error}^2}$. 
For instance, with $N=10$, selecting $R_\text{approx}=0.75$ yields a combined error of only $4\%$, whereas $R_\text{approx}=0.1$ increases the error to $19\%$. 
Although this represents a notable difference, the ability to capture and manipulate transmission zeros—central to the stabilisation mechanisms studied here—justifies the trade-off. 
From a modelling perspective, these errors remain modest, particularly given the new analytic and computational leverage gained within the DDE framework.
%
\par
%
A second limitation is that this study focuses on uniform gratings, which provide the most straightforward foundation for deriving a discretised-reflection model and serve as the canonical FBG design. 
The layered description, however, extends naturally to non-uniform gratings by allowing layer-dependent $r_k$, $t_k$, and $\Delta t_k$, with the analytical and continuation techniques developed here carrying over unchanged. 
This offers immediate opportunities for modelling gratings widely used in practice—such as Gaussian-apodised or phase-shift FBGs—while retaining the mathematical tractability of the discretised approach. 
%
\par
%
\begin{figure}
    \centering
    
    \includegraphics[width=0.95\linewidth]{Images/NR_Selection.pdf}
    
    \caption{Combined relative errors of FBG total reflectivity $R_\text{error}$ and delay $\tau_\text{error}$ as a function of the grating number $N$ and normalised refractive index variation $\dn$. 
    Level sets of constant total reflectivity $R_\text{approx}$ are overlayed in white. 
    In (a), the square marker at $(N,\dn) = (10, 0.07)$, corresponding to an $R_\text{approx} = 0.75$ has a combined error of 4\% while in (b), the marker at $(N,\dn) = (10, 0.0055)$, 
    corresponding to an $R_\text{approx} = 0.1$ has a combined error of 19\%.}
    
    \label{fig:NR_Selection}
\end{figure}
%
An especially promising direction is the extension of the framework to chirped gratings. 
Unlike uniform gratings, where each layer contributes an identical delay $\Delta t$ and reflection/transmission coefficient, a chirped grating introduces a frequency dependence into the reflection process. 
Different spectral components of the incoming field are reflected at different depths, making the effective reflection point frequency dependent \cite{erdogan1997fibre}. 
In a discretised model this translates into layer-specific delays $\Delta t_k(\omega)$ and reflectivities $r_k(\omega)$, which depend not only on the fixed grating profile but also on the instantaneous optical field itself. 
The resulting equations couple the laser dynamics to a state-dependent feedback path, placing the system within the broader class of state-dependent delay differential equations (sd-DDEs). 
%
\par
%
This represents both a challenge and an opportunity. 
On the one hand, sd-DDEs are more difficult to analyse, as continuation tools must be adapted to handle bifurcations when delays themselves depend on the state. 
On the other hand, incorporating chirped gratings would capture a far richer set of physical effects central to practical applications, including broadband dispersion compensation, tunable feedback stabilisation, and advanced schemes for chaos control \cite{wang2017time,wang2019key,wang2023critical}. 
In particular, the ability to tailor how different frequency components interact with the grating in real time may enable selective stabilisation of higher-order modes or controlled shaping of chaotic attractors for secure communication systems. 
The discretised-reflection formulation thus offers a natural stepping stone toward bridging idealised uniform gratings and the complex non-uniform designs used in state-of-the-art optoelectronic devices.
%
\par
%
In summary, the discretised-reflection formulation opens a broad outlook beyond the technical advances demonstrated here. 
By embedding FBG feedback squarely within a DDE framework, it provides a practical and extensible tool for connecting grating design with laser dynamics in a way that is both analytically transparent and computationally efficient. 
This enables systematic exploration of stability mechanisms, the organisation of nonlinear regimes, and the shaping of chaotic behaviour, with immediate relevance to applications ranging from secure communications to high-precision sensing. 
More broadly, the approach offers a pathway toward unifying idealised models and realistic device architectures, supporting both fundamental investigations of delay-coupled lasers and the design of next-generation photonic systems where stability, controllability, and harnessed chaos are central requirements.

